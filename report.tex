\documentclass[a4paper,12pt]{article}
% !TEX root = ./report.tex

\usepackage[numbers,comma]{natbib}

\usepackage{enumerate}		%For getting different types of lists like a) II) and so forth.

\usepackage[T1]{fontenc}
\usepackage[utf8]{inputenc}	%For norwegian letters and UTF8 encoding support
\usepackage{lastpage}		%For the command \pageref{lastpage}

\usepackage[english]{babel}	%For filltext in conjunction with blindtext
\usepackage{blindtext}		%For filltext (see comment for babel)

\usepackage{xargs}	% Use more than one optional parameter in a new commands
\usepackage[colorinlistoftodos,prependcaption]{todonotes} %For \todo
\newcommandx{\thiswillnotshow}[2][1=]{\todo[disable,#1]{#2}}
\newcommandx{\unsure}[2][1=]{\todo[linecolor=red,backgroundcolor=red!25,bordercolor=red,#1]{#2}}
\newcommandx{\change}[2][1=]{\todo[linecolor=blue,backgroundcolor=blue!25,bordercolor=blue,#1]{#2}}
\newcommandx{\info}[2][1=]{\todo[linecolor=OliveGreen,backgroundcolor=OliveGreen!25,bordercolor=OliveGreen,#1]{#2}}
\newcommandx{\improvement}[2][1=]{\todo[linecolor=Plum,backgroundcolor=Plum!25,bordercolor=Plum,#1]{#2}}

\usepackage[colorlinks]{hyperref}	%For \url{}
\hypersetup{linkcolor=black}
\hypersetup{citecolor=black}
\hypersetup{urlcolor=black}

\usepackage{datetime}		%For \newdateformat
\newdateformat{CurMonth}{\monthname[\THEMONTH],~\THEYEAR}

\makeatletter


\begin{document}

\title{The effects and consequences of Moore's Law in the 2000s and going forward on the IT~business.}
\author{Christian~Chavez}

\maketitle

Moore's Law~\cite{Moore:2000:CMC:333067.333074} originates from Moore's prediction back in 1965 with the consequence that the sizes of circuits in microchips would halve every 18 months, resulting in price drops on computer equipment with every corresponding new generation of microchips (containing the double amount of transistors, thus easing the increase of clock speed on the microchips).

In the 1960-1990 decades, as the hardware almost doubled the frequency of its processing cycles, enabling software written for earlier generations of hardware to nearly double with the next generation hardware, with no effort required from the software developers.

However, this did not last for long, as the cost of new factories/hardware needed to produce computer hardware at smaller sizes\footnote{Following the prediction of Moore's Law.} got prohibitively more and more expensive. As a matter of fact, nanometer hardware production facilities began costing exponentially more for each hardware generation, far outweighing the doubled gain of processor frequency/amount of transistors on the microchip. This is became known as Moore's second Law, as well as Rock's Law~\cite{Ross:2003:EC:965307.965316}.

The limit imposed by nanometer technology, not to mention its cost, is also why the end of Moore's Law has been prophesized, announced, and declared many a time over the last decades.

Thus,  enabled the rise of the computer industry, becoming the giant we know it to be today.
The repeating cycle of better hardware, at lower cost, enabled hardware companies like Intel and AMD to persevere.
Others tried the same, but companies such as Nvidia, IBM, and Sun did not manage to achieve the same success, while others again died out in the competition.
Yet, hardware companies were not the only ones to profit from Moore's Law.
The IT industry of \emph{today} is known for its software giants, such as Google, Microsoft, Linux.

Thus, this is one of the reasons why software production was comparatively cheaper pre-2000, compared to the cost of producing a comparably sized software project post-2000.
The software developers pre-2000 did not need to account for the myriad of different hardware architectures that exist today (let alone software- architectures/libraries/languages), and their code performed noticeably faster with each new generation of hardware.

Post-2000 consumers, in private and business sectors alike, rarely, if ever, notice an improvement in the required run-time for the same code on a new generation of hardware.

While \todo{insert references}some have predicted its end to occur in the last two decades or so, others~\todo{insert more references}\cite{Moore40,Moore50} still maintain that it carries well with age. What cannot be denied is that Moore's Law made a good match with the exponential growth of clock speed on affordable electronics in the years from the late 1960s till late 1990s.

The ones \todo{figure out what to write here}\textit{\textbf{X}} who have called for the end of Moore's law, often list the following reasons for its come/coming demise:

\begin{enumerate}[1)]
	\item The
\end{enumerate}

\clearpage
\bibliographystyle{unsrtnat}
\bibliography{bibliography}

\end{document}
