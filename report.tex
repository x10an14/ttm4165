\documentclass[a4paper,12pt]{article}
% !TEX root = ./report.tex

\usepackage[numbers,comma]{natbib}

\usepackage{enumerate}		%For getting different types of lists like a) II) and so forth.

\usepackage[T1]{fontenc}
\usepackage[utf8]{inputenc}	%For norwegian letters and UTF8 encoding support
\usepackage{lastpage}		%For the command \pageref{lastpage}

\usepackage[english]{babel}	%For filltext in conjunction with blindtext
\usepackage{blindtext}		%For filltext (see comment for babel)

\usepackage{xargs}	% Use more than one optional parameter in a new commands
\usepackage[colorinlistoftodos,prependcaption]{todonotes} %For \todo
\newcommandx{\thiswillnotshow}[2][1=]{\todo[disable,#1]{#2}}
\newcommandx{\unsure}[2][1=]{\todo[linecolor=red,backgroundcolor=red!25,bordercolor=red,#1]{#2}}
\newcommandx{\change}[2][1=]{\todo[linecolor=blue,backgroundcolor=blue!25,bordercolor=blue,#1]{#2}}
\newcommandx{\info}[2][1=]{\todo[linecolor=OliveGreen,backgroundcolor=OliveGreen!25,bordercolor=OliveGreen,#1]{#2}}
\newcommandx{\improvement}[2][1=]{\todo[linecolor=Plum,backgroundcolor=Plum!25,bordercolor=Plum,#1]{#2}}

\usepackage[colorlinks]{hyperref}	%For \url{}
\hypersetup{linkcolor=black}
\hypersetup{citecolor=black}
\hypersetup{urlcolor=black}

\usepackage{datetime}		%For \newdateformat
\newdateformat{CurMonth}{\monthname[\THEMONTH],~\THEYEAR}

\makeatletter


\begin{document}

\title{The effects and consequences of Moore's Law post-2000.}
\author{Christian~Chavez}

\maketitle

\section*{Moore's Law}

Moore's Law~\cite{Moore:2000:CMC:333067.333074} originates from Moore's prediction back in 1965 with the consequence that the sizes of circuits in microchips would halve every 18 months (doubling the amount of transistors), resulting in price drops on computer equipment with every corresponding new generation of microchips.
The doubling of transistors had a consequence of permitting a near doubling in processor cycles per new hardware generation.
In the 1960-1990 decades, with each generation of new hardware, the hardware almost doubled the frequency of its processing cycles, enabling software written for earlier generations of hardware to nearly halve their execution speed with the next generation hardware, with no effort required from the software developers.

However, this did not last for long, as the cost of new factories/hardware needed to produce computer hardware at smaller sizes\footnote{Following the prediction of Moore's Law.} got prohibitively expensive.
Eventually, hardware production facilities began costing exponentially more for each hardware generation, far outweighing the doubled gain of processor frequency/amount of transistors on the microchip.

Later, this became known as Moore's second Law\footnote{For the rest of this Essay, Moore's First Law will always be referred to as ``Moore's Law'' unless otherwise specified.}, as well as Rock's Law~\cite{Ross:2003:EC:965307.965316}.
This limit imposed by nanometer technology, not to mention its cost, is also why the end of Moore's Law has been prophesized, announced, and declared many a time over the last decades.

\section*{Its influence on the IT industry}

Moore's Law enabled the rise of the computer industry, becoming the giant we know it to be today.
The repeating cycle of better hardware, at lower cost for the consumer, enabled hardware companies like Intel\footnote{Intel bet their business plans on Moore's Law prediction's holding true throughout several decades\cite{IntelMoore'sLawIn10nm,Moore40,Moore50}.} to persevere.

Others tried the same, but companies such as AMD, Nvidia, IBM, and Sun did not achieve the same success, while others again perished in the wake of their competition.
Yet, hardware companies were not the only ones to profit from Moore's Law.
The IT industry of \emph{today} is known for its software giants, such as Google, Microsoft, Apple.

One can safely say that the effect(s) (or conversely; consequences) of Moore's Law has had a massive impact on the IT industry.
Nothing of the industry, as we know it today, would exist without the ability of our current technology to cram so many transistors into Integrated Circuits (ICs).
We have the Internet as we know it today, in part due to the now simple hardware requirements for a router, being within reach of the average private consumer.

\section*{Its influence on our Markets and Society}

With Rock's Law (Moore's second Law) following up Moore's Law, microchip fabrication companies want to maximize the profits they may get out of their fabrication plants.
Thus, older generation computer hardware has become ubiquitous, having a big impact on any society on Earth.
From the rich Western societies, `till the poorest villages in Africa, the spread of the Internet, not to mention computerized appliances (of which the smart-phone is a great late example of), the consequential effects of Moore's Law has transformed our world as a whole.
No society (that we know of) has been left untouched by Moore's Law through the passage of time.

With the IT industry, having gotten such a foothold as a pillar of today's societies, it is a natural consequence that both local and global markets are affected by (maybe even heavily invested in) the IT industry\footnote{The top stocks of the stock markets of today hold names such as Google, Facebook, Intel, and Apple\cite{ITstockMarkets}.}.
Any rise and fall in the IT/Tech industry can have immense global consequences, and many draw correlations between the stock market climate of today, and the stock market climate of 1999, in fear of a another tech stock market bubble\cite{TechBubbleBursting}.

Social network companies, like Facebook, have an extra foothold\cite{FacebookInternetAfrica} in our societies in this day and age.
With the information people freely give (and some information Facebook attempts to force people to give\cite{FacebookRealNamePolicyConsequences}), the consequences if the information held by the company ended up in the wrong hands, becoming maliciously used, could spell societal disaster.

Unsuspecting employees, friends, family members have been fired, ostracized, and even persecuted for releasing information about their sexual orientation, gender, ethnicity, as well as the more understandable slurs they may spout.

\section*{The continuing consequences and effects of Moore's Law going forward}

Post-2000, Moore's Law has heavily affected the IT industry, our societies, and markets, as we've discussed.
With the diminishing returns of having new hardware executing the same code noticeably more efficient, the IT industry has shown a propensity to develop software which needs \emph{more} resources per hardware generation than the generation is able to supply.

This has later been coined Wirth's Law\cite{TheGreatMoore'sLawCompensator}, which states ``Software gets slower faster than hardware gets faster''\footnote{This being one of the bigger reasons for push towards multi-core parallelism, as it's implementation on the hardware side became cheaper than keeping up with the single-core, single-thread performance expectations for each new hardware generation.}.
The software industry's move from Waterfall methodology to Agile can be seen as a consequence of this law.

The IT industry, having become so well established as one of the consequences of Moore's Law, now contains a plethora of smaller companies who might become the next Intel, Google, Apple, or Microsoft in technologies like embedded devices, operating systems, compilers, web browsers, and etc.
Companies like Netflix, Spotify, the Raspberry PI Foundation, Facebook, and Silicon Labs all have the origins of their current, well-known form in the decade of 2010, 45 years since Moore's Law was first proposed.
While one can never predict the future with any accuracy, it's a safe statement that the consequences Moore's Law still has some effects today, if only due to the movement of the IT into the cloud\cite{Moore'sLawCloudPricing}.

And who's to say that not-yet invented applications deriving from the consequences/effects of Moore' Law cannot help with more basal needs\cite{Moore'sLawTurns50}, such as overcrowding and water supply?
Yet, Moore himself\cite{MooreSpeculates} does not think that his Law will last forever.
He even states that he thought extrapolating 10 years into the future (which was what his original paper did) was ``pretty wild'', and that it took him two decades before he managed to utter ``Moore's Law'' himself with a straight face.

So while even the famous Gordon E. Moore himself does not dare to attempt another prediction at what comes next, there's always the possibility that a new innovation either brings forth a new Law, or enables the continuation of Moore's Law\cite{FutureForTransistor}.

\bibliographystyle{unsrtnat}
\bibliography{bibliography}

\end{document}
